\title{Waterfall Optimization}
\documentclass[12pt]{article}
\usepackage{amsfonts}
\begin{document}
\date{}

\maketitle
	\section*{Description}
		We are ordering a sequence of N unique events: $A_{1}...A_{N}$.\\
		We are given a time budget $T_{max}$.\\
		Each event may either succeed or fail.\\
		Each event $A_{i}$ may succeed with some probability $P(A_{i})$.\\
		It will fail with probability $1-P(A_{i})$.\\
		Each event if successful will provide reward $R(A_{i})$.\\
		Each event if successful will cost time $T_{i}^{s}$.\\
		Each event if unsuccessful will cost time $T_{i}^{f}$.\\
		$T_{i}^{s}$, $T_{i}^{f}$ are Gaussian random variables with known mean.\\
		Assume $R(A_{i})$ and $P(A_{i})$ is known.\\
		The ordering will be evaluated from start to finish until the first successful event.
		We must return an ordering of the events such that the expected value of the evaluation of the ordering is maximal.\\
	\newpage
	\section*{Optimization}
		$Max_{(A_1...A_N)}\sum_{i} R_{A_{i}}*P_{A_{i}}*\prod_{j<i} (1-P_{A_{j}})*P(T_{i}^{s}+\sum_{j<i}(T_{j}^{f})<T_{Max})$.\\
		$A_{i} \in \mathbb{Z}$\\
		$1 \le A_{i} \le N$\\
		$AllDiff(A_i)$\\\\
        \noindent
        Where $P(T_{i}^{s}+\sum_{j}(T_{i}^{f})<T_{Max})$ is evaluated as a cumulative distribution function with 
        mean $T_{i}^{s}+\sum_{j<i}(T_{j}^{f})$ and known variance.\\

        \noindent
        $CDF(x,\mu,\sigma) = \frac{1}{2}[1+erf(\frac{x-\mu}{\sqrt{2}})]$ \\
        Where erf is the error function that can be estimated with:\\
        $erf(w) = 1 - (a_1t^1+a_2t^2+a_3*t^3+a_4*t^4+a_5*t^5)e^{-w^2}$\\\\
        Where $t = \frac{1}{1+pw},$\\
        $p = 0.3275911,$\\
        $a_1 = 0.254829592,$\\
        $a_2 = −0.284496736,$\\
        $a_3 = 1.421413741,$\\
        $a_4 = −1.453152027,$\\
        $a_5 = 1.061405429$ 
	\newpage
    \section*{Optimization Method}
    To perform the optimization we apply a stochastic local search (SLS) on the solution space in combination with brufe force search for small sized problem instances.
    Sequences of size less than 9 are optimized by a direct brute force DFS (depth first search). The utility of each solution is evaluated directoly as described in the previous section.
    Sequences of size 9 or greater are evaluated by SLS. In particular we borrow the idea of a decreasing neighborhood size as candidate solutions become more and more optimal. 
    The search method proceeds as follows: A base candidate solution is generated by sorting on each of the columns in the input data (generating a best to worst order at each sort). 
    This order is then searched on in a windowed scheme. A window size is determined and the local seach proceeds for the duration of the window. If after the duration no better solution has 
    been found than the best found at the start of the window seach terminates and the best found solution is returned otherwise search proceeds for anothe window. Within the local search for 
    each window the incumbent solution (the best found thus far) is cloned, it is modified into an alternative solution in the neighborhood of the candidates solution. The utility of each 
    such candidate solution is then evaluated, and if the utility is greater than that of the incumbent the candidate solution replaces the incumbent solution and becomes the incumbent. Each time 
    an incumbent solution is replaced, the difference in utility between the replaced incumbent and the replacing candidate is recorded and retained. There are 2 methods for generating a solution 
    in the neighborhood of an incumbent. Firstly, 2 points are chosen from within the incumbent and the flights in those two positions are swapped. Alternatively, a full random shuffle is performed on
    the candidate. The probability of using the second method is inversly proportional to the size of the largest utility improvement (caused by replacing an incumbent with a candidate) within 
    the current window. In other words, as the added utility of improved incumbent solution decreases the chances of a random shuffle grow. Finally, if the requested waterfall size is smaller than 
    the full number of possible flights for a placement a small modification to this algorithm is made. The initial set is partitioned into two sets. The first set is considered the waterfall of 
    the required size, and the next is the set of all remaining flights for that placement. In the bootstrap phase the first candidate solution in such a small sized waterfall will consist of the 
    N "best" (as determined by sorting) flights. In the SLS component, when a neighborhood solution is needed it is generated by swapping a flight from the candidate with a flight from the remaining
    flights for the placement. In other respects the two solution methods are identical.

\end{document}
